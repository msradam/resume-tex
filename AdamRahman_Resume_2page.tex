% !TEX program = xelatex
%
% Adam Rahman's Resume Template
%

\documentclass[letterpaper, 11pt]{article}

\usepackage{inputenc}
\usepackage{fontspec}
\usepackage[margin=0.75in]{geometry}
\usepackage{titlesec}
\usepackage{colortbl}
\usepackage{titling}
\usepackage[rm]{roboto}
% \usepackage{libertine}
% \usepackage[sfdefault]{equity}
\usepackage{xifthen}
\usepackage{hyperref}
\usepackage{sqrcaps}
\pagenumbering{gobble} % A quick hack to get rid of the page number 

\hypersetup{
    colorlinks=true,
    linkcolor=blue,
    filecolor=magenta,      
    urlcolor=blue,
}

\author{Adam M. Rahman}

% 
% Title Command
% Generates resume header
%
% \renewcommand{\maketitle}{
% 	\hspace{.125\textwidth}
% 	\begin{minipage}[t]{.75\textwidth}
%     \begin{center}
%         \fontsize{16pt}{15pt}\selectfont\bfseries \theauthor \\
%         \fontsize{10pt}{15pt}\selectfont\sf
%         \href{mailto:munawaradamr@gmail.com}{munawaradamr@gmail.com} · (347) 593 7525 \\
%         Portfolio: \href{http://adamr.io}{adamr.io} · 
%         LinkedIn: \href{https://www.linkedin.com/in/adamsrahman/}{adamsrahman} · 
%         Github: \href{https://github.com/msradam}{msradam}  \\ 
%         \end{center}
%     \end{minipage}}

\renewcommand{\maketitle}{
	\hspace{.125\textwidth}
	\begin{minipage}[t]{.75\textwidth}
    \begin{center}
        \fontsize{15pt}{15pt}\selectfont\bfseries \theauthor \\
        \fontsize{10pt}{15pt}\selectfont\normalfont
        \vspace{-0.5mm}
        \href{mailto:munawaradamr@gmail.com}{msrahmanadam@gmail.com} · (347) 593 7525 \\
        \vspace{-0.5mm}
        Portfolio: \href{https://adamr.io}{adamr.io} · 
        LinkedIn: \href{https://www.linkedin.com/in/adamsrahman/}{adamsrahman} · 
        GitHub: \href{https://github.com/msradam}{msradam}  \\ 
        \end{center}
    \end{minipage}}


% \renewcommand{\familydefault}{\sfdefault}

%
% Defining separate entry commands for Education, Experience, Projects, and Skills
%
\newcommand{\eduentry}[4]{
        \begin{minipage}[b]{0.5\textwidth}
        \raggedright
        \bf\large #2
        \end{minipage}%
        \begin{minipage}[b]{0.5\textwidth}
        \raggedleft
        \bf #1
        \end{minipage}

    % \begin{minipage}[t]{.15\linewidth}
    % \hfill \textsc{#1}
    % \end{minipage}
    % \hfill\vline\hfill
    \begin{minipage}[t]{.80\linewidth}
    \vspace{-3mm}
    \textit{#3} \small{#4}
    \end{minipage}\\
    \vspace{1mm}
    }

% Experience Schema:
% {start date} {end date} {company} {position} {description}
\newcommand{\expentry}[5]{

    \begin{minipage}[b]{0.5\textwidth}
        \raggedright
        \bf\large #3
        \end{minipage}%
        \begin{minipage}[b]{0.5\textwidth}
        \raggedleft
        \bf {#1} -- {#2}
        \end{minipage}

    % \begin{minipage}[t]{.15\linewidth}
    % \hfill \textsc{#1} \\
    % \hfill \hspace*{5pt}\hfill --- \textsc{#2}
    % \end{minipage}
    % \hfill\vline\hfill
    \begin{minipage}[t]{\linewidth}
    \vspace{-3mm}
    #4
    \vspace{-1.75mm}
    \small{#5}
    \end{minipage}\\
    \vspace{1mm}
    }

 \newcommand{\projectentry}[4]{

    \begin{minipage}[b]{0.5\textwidth}
        \raggedright
        \bf #2
        \end{minipage}%
        \begin{minipage}[b]{0.5\textwidth}
        \raggedleft
        \bf #1
        \end{minipage}

    % \begin{minipage}[t]{.15\linewidth}
    % \hfill \textsc{#1} \\
    % \hfill \hspace*{5pt}\hfill --- \textsc{#2}
    % \end{minipage}
    % \hfill\vline\hfill
    \begin{minipage}[t]{\linewidth}
    \vspace{-3mm}
    \small #3
    \vspace{-1.75mm}
    \small{#4}
    \end{minipage}\\
    \vspace{1mm}
    }



    \newcommand{\awardentry}[4]{
        \begin{minipage}[t]{.15\linewidth}
        \hfill \textsc{#1}
        \end{minipage}
        \hfill\vline\hfill
        \begin{minipage}[t]{.80\linewidth}
        {\bf#2}
        \\ #3 
        \vspace{-1.5mm}
        \small{#4}
        \end{minipage}\\
        \vspace{.10cm}
        }           
    


\newcommand{\skillentry}[2]{
    \begin{minipage}[t]{.15\linewidth}
        \hfill \textsc{#1}
        \end{minipage}
        \hfill\vline\hfill
        \begin{minipage}[t]{.80\linewidth}
        \small{#2}
        \end{minipage}\\
    }


% Title and line formatting
\titleformat{\part}{\Huge\scshape\filcenter}{}{1em}{}
\titleformat{\section}{\scshape\bfseries\raggedright}{}{0.0em}{}[{\titlerule[0.75pt]}]
\titlespacing{\section}{0pt}{3pt}{7pt}
\titleformat{\subsection}{\large\bfseries\centering}{}{0em}{\underline}%[\rule{3cm}{.2pt}]
\titlespacing{\subsection}{0pt}{7pt}{7pt}
\let\lineheight\baselineskip
\setlength{\parindent}{0in}


\begin{document}
    \maketitle
    \vspace{.075cm}


    \section{education}
    \eduentry{Sept. 2016 -- May 2019}
    {Wesleyan University}
    {B.A. Computer Science, Theater; GPA: 3.36/4.0}
    {\\ Graduated May 2019; Patricelli Center Fellow; Deans' List Spring 2017\\
    \textbf{Relevant Coursework:} Algorithms \& Complexity, Computer Networks, Design of \\ Programming Languages, Randomized Algorithms, Proseminar in Audiovisual\\Machine Learning, Patricelli Center Fellowship}

    \eduentry{Sept. 2012 -- June 2016}
    {Stuyvesant High School}
    {High School Diploma}
    {\\ Graduated June 2016; Arista National Honor Society; Varsity Lincoln-Douglas Debate}

    \section{experience}
    \expentry{July 2020}
    {Present}
    {IBM}
    {Staff Software Developer}
    {
        \begin{itemize}
          \setlength\itemsep{-0.5mm}
            \item Develops, executes, and maintains continuous integration and regression tests
            for IBM Z mainframe software components, including Docker containers running via IBM z/OS Container
            Extensions (zCX), IBM zCX Foundation for Red Hat OpenShift,  and z/OS Management Facility RESTful APIs
            \item Orchestrates test automation infrastructure with Ansible and Jenkins to execute and monitor
            health checks, load/stress tests, and customer workloads across a series of on-premise mainframe test hardware and IBM Wazi as a Service Virtual Machines on IBM Cloud
            \item Develops and ports test utilities with Metal C, z/OS High Level Assembler, and Golang to exploit and test z/OS services and run open-source tools such as Grafana k6 and Locust natively
            \item Conducts daily monitoring and maintenance of mainframe test systems,
            including mounting filesystems, installing products such as Conda, Node.js,
            and Z Open Automation Utilities, and troubleshooting system dumps and traces in coordination
            with other developers and testers
        \end{itemize}
        
    }

    
    \expentry{Sept. 2019}
    {Nov. 2019}
    {Buildly}
    {Software Development Intern}
    {
        \begin{itemize}
          \setlength\itemsep{-0.5mm}
          \item Worked in a remote team to implement bugfixes, write unit tests, and refactor features for Buildly's Django backend in order to connect multiple microservices to a single endpoint, manage several databases, and incorporate self-documenting API specifications with Swagger UI
          \item Maintained end-to-end test suite for Buildy's frontend and backend with the Robot framework
          \item Communicated with CEO and CTO daily about product roadmap and development guidelines
        \end{itemize}
        
    }

    \expentry{Sept. 2019}
    {Dec. 2019}
    {Kite.com}
    {Python Code Curator}
    {
        \begin{itemize}
          \setlength\itemsep{-0.5mm}
          \item Composed over 30 code samples with accompanying tutorials in Markdown, adhering to Python best practices, to answer the most frequently asked programming questions
          \item Coordinated remotely with teammates across the U.S. to review and verify code samples
         \end{itemize}
        
    }

    \expentry{Dec. 2018}
    {Feb. 2019}
    {UNICEF}
    {Software Development Intern}
    {
        \begin{itemize}{\leftmargin=0.5em \itemindent=0em}
          \setlength\itemsep{-0.5mm}
          \item Implemented pipelines to retrieve road networks per country and compute distances between school and health facility coordinates extracted from large geospatial datasets 
          \item Optimized algorithm performance to improve computation speed by hundreds of times across thousands of geospatial points by researching and incorporating fast Python libraries
          \item Demoed solution to team members and published \href{https://github.com/unicef/magicbox-download-roads}{road retrieval code} to UNICEF's official GitHub
          \item Provisioned a \href{https://hub.docker.com/r/msradam/magicbox-tools}{Docker image} bundled with pipeline code and requisite libraries intended for use by Magicbox developers and data scientists
        \end{itemize}
        
    }

    \expentry{June 2018}
    {August 2018}
    {Kurani Architecture}
    {Software Development Intern}
    {\begin{itemize}
          \setlength\itemsep{-0.5mm}
          \item Integrated Raspberry Pi sensor data and scraped content from TedEd and Khan Academy to prototype 'Aristotle', a responsive learning feed for classrooms and smart devices
          \item Collaborated with CEO to brainstorm IoT implementation in learning spaces 
        \end{itemize}
    }

    \expentry{Sept. 2017}
    {Feb. 2019}
    {Wesleyan Local Food Co-Op}
    {Volunteer Programmer}
    {\begin{itemize}
          \setlength\itemsep{0.1mm}
          \item Implemented web and command-line utilities for Wesleyan's Local Food Cooperative, a student-run group that brings sustainable locally sourced food to the Wesleyan community
          \item Retrieved Qualtrics survey data, computed share prices for participants, and outputted spreadsheets for co-op coordinators to determine total shares and money owed per-person
        \end{itemize}
    }



    \section{projects}
        \projectentry{March 2025}
        {AskStreets: Analyzing Street Networks With AI}
        {ArangoDB, NetworkX, LangGraph | \href{https://github.com/msradam/askstreets}{github.com/msradam/askstreets}}
        {\begin{itemize}
            \setlength\itemsep{-0.5mm}
            \item Developed an agentic AI app to parse natural language user queries, process graph structures built from OpenStreetMap, and execute graph algorithms to generate and visualize insights into street networks
            \item Second Place Winner of ArangoDB and Nvidia's "Building the Next-Gen Agentic App with GraphRAG \& NVIDIA cuGraph" Hackathon, March 2025

        \end{itemize}
        } 

        \projectentry{March 2020}
        {HeatTweets: Geocoding Fire Incidents in NYC}
        {R, Python, KeplerGl | \href{https://github.com/msradam/NYCFireData}{github.com/msradam/NYCFireData}}
        {\begin{itemize}
            \setlength\itemsep{-0.5mm}
            \item Worked in a team to scrape, geocode, and map fire incidents from Twitter using R, Python, and KeplerGl
            \item Presented at NYC's Open Data Week Hackathon: "Can Data Start a Movement?" at Queens College, NY
        \end{itemize}
        }



        \projectentry{August 2019}
        {Carceral Contagion: Incarceration Simulation}
        {Mesa, NetworkX, Python | \href{https://github.com/msradam/carceral-contagion}{github.com/msradam/carceral-contagion}}
        {\begin{itemize}
            \setlength\itemsep{-0.5mm}
            \item Simulated an epidemiological model of U.S. mass incarceration based on historical data, illustrating racial disparities and the infectious nature of imprisonment
            \item Modelled the population network with NetworkX graph data structures and implemented an interactive frontend with the Mesa agent-based simulation library
            \item Project prototype earned Social Good Award + 2nd Place Crowd Favorite at Wellesley Hacks, Nov. 2017
        \end{itemize}
        }

        \projectentry{July 2019}
        {Geospatial Routing API}
        {Scikit-learn, Django, Docker, OpenAPI | \href{https://github.com/msradam/geospatial-routing-api}{github.com/msradam/geospatial-routing-api}}
        {\begin{itemize}
            \setlength\itemsep{-0.5mm}
            \item Expanded upon my UNICEF internship project by implementing a microservice with API endpoints to compute straight-line and routed distances between geographic coordinates and retrieve roads as graph data structures, with a Django backend and Swagger UI to view endpoint documentation
        \end{itemize}
        }


        \projectentry{June 2019}
            {RESP: Responsive Emotional Support Protocols}
            {Node.js, Express, React, MongoDB | \href{https://github.com/msradam/resp}{github.com/msradam/resp}}
            {\begin{itemize}
                \setlength\itemsep{-0.5mm}
                \item Worked in a team of software developers and UX designers to implement a React app with Typeform webhooks to check-in natural disaster survivors, store them in a MongoDB database, and query the open-source Healthsites.io API for nearby medical facilities
                \item Winner of the IBM Call for Code Challenge at AngelHack Manhattan, June 2019
            \end{itemize}
            }
            
    \section{technical skills}
    \skillentry{languages}{Python, JavaScript (ES6+), Java, C, Bash, Markup: HTML5, CSS3, Markdown}
    \skillentry{concepts}{RESTful API Design/Integration, MVC Architecture, Agile}
    \skillentry{frameworks}{Docker, Kubernetes, Django, Flask, Express, React, Robot}
    \skillentry{tools}{Git, GitHub, Jenkins, IBM Cloud, Ansible, Locust, Grafana k6}
    \skillentry{data}{PostgreSQL, GraphQL, MongoDB, IBM Db2}
    \skillentry{testing}{Unit, Integration, Regression, End-to-End, Load/Stress}

\end{document}

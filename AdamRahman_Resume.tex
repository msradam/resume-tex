% !TEX program = xelatex
%
% Adam Rahman's Resume Template
%

\documentclass[letterpaper, 10.5pt]{article}

\usepackage{inputenc}
\usepackage{fontspec}
\usepackage[margin=0.375in]{geometry}
\usepackage{titlesec}
\usepackage{colortbl}
\usepackage{titling}
\usepackage[rm]{roboto}
% \usepackage{libertine}
% \usepackage[sfdefault]{equity}
\usepackage{xifthen}
\usepackage{hyperref}
\usepackage{sqrcaps}
\pagenumbering{gobble} % A quick hack to get rid of the page number 

\hypersetup{
    colorlinks=true,
    linkcolor=blue,
    filecolor=magenta,      
    urlcolor=blue,
}

\author{Adam M. Rahman}

% 
% Title Command
% Generates resume header
%
% \renewcommand{\maketitle}{
% 	\hspace{.125\textwidth}
% 	\begin{minipage}[t]{.75\textwidth}
%     \begin{center}
%         \fontsize{16pt}{15pt}\selectfont\bfseries \theauthor \\
%         \fontsize{10pt}{15pt}\selectfont\sf
%         \href{mailto:munawaradamr@gmail.com}{munawaradamr@gmail.com} · (347) 593 7525 \\
%         Portfolio: \href{http://adamr.io}{adamr.io} · 
%         LinkedIn: \href{https://www.linkedin.com/in/adamsrahman/}{adamsrahman} · 
%         Github: \href{https://github.com/msradam}{msradam}  \\ 
%         \end{center}
%     \end{minipage}}

\renewcommand{\maketitle}{
	\hspace{.125\textwidth}
	\begin{minipage}[t]{.75\textwidth}
    \begin{center}
        \fontsize{15pt}{15pt}\selectfont\bfseries \theauthor \\
        \fontsize{10pt}{15pt}\selectfont\normalfont
        \vspace{-0.5mm}
        \href{mailto:munawaradamr@gmail.com}{msrahmanadam@gmail.com} · (347) 593 7525 \\
        \vspace{-0.5mm}
        Portfolio: \href{https://adamr.io}{adamr.io} · 
        LinkedIn: \href{https://www.linkedin.com/in/adamsrahman/}{adamsrahman} · 
        GitHub: \href{https://github.com/msradam}{msradam}  \\ 
        \end{center}
    \end{minipage}}


% \renewcommand{\familydefault}{\sfdefault}

%
% Defining separate entry commands for Education, Experience, Projects, and Skills
%
\newcommand{\eduentry}[4]{
        \begin{minipage}[b]{0.5\textwidth}
        \raggedright
        \bf\large #2
        \end{minipage}%
        \begin{minipage}[b]{0.5\textwidth}
        \raggedleft
        \bf #1
        \end{minipage}

    % \begin{minipage}[t]{.15\linewidth}
    % \hfill \textsc{#1}
    % \end{minipage}
    % \hfill\vline\hfill
    \begin{minipage}[t]{.80\linewidth}
    \vspace{-3mm}
    \textit{#3} \small{#4}
    \end{minipage}\\
    \vspace{1mm}
    }

% Experience Schema:
% {start date} {end date} {company} {position} {description}
\newcommand{\expentry}[5]{

    \begin{minipage}[b]{0.5\textwidth}
        \raggedright
        \bf\large #3
        \end{minipage}%
        \begin{minipage}[b]{0.5\textwidth}
        \raggedleft
        \bf {#1} -- {#2}
        \end{minipage}

    % \begin{minipage}[t]{.15\linewidth}
    % \hfill \textsc{#1} \\
    % \hfill \hspace*{5pt}\hfill --- \textsc{#2}
    % \end{minipage}
    % \hfill\vline\hfill
    \begin{minipage}[t]{\linewidth}
    \vspace{-3mm}
    #4
    \vspace{-1.75mm}
    \small{#5}
    \end{minipage}\\
    \vspace{1mm}
    }

 \newcommand{\projectentry}[4]{

    \begin{minipage}[b]{0.5\textwidth}
        \raggedright
        \bf #2
        \end{minipage}%
        \begin{minipage}[b]{0.5\textwidth}
        \raggedleft
        \bf #1
        \end{minipage}

    % \begin{minipage}[t]{.15\linewidth}
    % \hfill \textsc{#1} \\
    % \hfill \hspace*{5pt}\hfill --- \textsc{#2}
    % \end{minipage}
    % \hfill\vline\hfill
    \begin{minipage}[t]{\linewidth}
    \vspace{-3mm}
    \small #3
    \vspace{-1.75mm}
    \small{#4}
    \end{minipage}\\
    \vspace{1mm}
    }



    \newcommand{\awardentry}[4]{
        \begin{minipage}[t]{.15\linewidth}
        \hfill \textsc{#1}
        \end{minipage}
        \hfill\vline\hfill
        \begin{minipage}[t]{.80\linewidth}
        {\bf#2}
        \\ #3 
        \vspace{-1.5mm}
        \small{#4}
        \end{minipage}\\
        \vspace{.10cm}
        }           
    


\newcommand{\skillentry}[2]{
    \begin{minipage}[t]{.15\linewidth}
        \hfill \textsc{#1}
        \end{minipage}
        \hfill\vline\hfill
        \begin{minipage}[t]{.80\linewidth}
        \small{#2}
        \end{minipage}\\
    }


% Title and line formatting
\titleformat{\part}{\Huge\scshape\filcenter}{}{1em}{}
\titleformat{\section}{\scshape\bfseries\raggedright}{}{0.0em}{}[{\titlerule[0.75pt]}]
\titlespacing{\section}{0pt}{3pt}{7pt}
\titleformat{\subsection}{\large\bfseries\centering}{}{0em}{\underline}%[\rule{3cm}{.2pt}]
\titlespacing{\subsection}{0pt}{7pt}{7pt}
\let\lineheight\baselineskip
\setlength{\parindent}{0in}


\begin{document}
    \maketitle
    \vspace{.075cm}


    \section{education}
    \eduentry{Sept. 2016 -- May 2019}
    {Wesleyan University}
    {B.A. Computer Science, Theater; 3.4/4.0}
    {\\ Graduated May 2019; Patricelli Center Fellow; Deans' List Spring 2017\\
    \textbf{Relevant Coursework:} Randomized Algorithms, Algorithms \& Complexity, Functional Programming, \\ Computer Networks, Design of Programming Languages, Proseminar in Audiovisual Machine Learning}

    \section{work experience}
    \expentry{July 2020}
    {Present}
    {IBM}
    {Staff Software Developer}
    {
        \begin{itemize}
          \setlength\itemsep{-0.5mm}
            \item Develop, execute, and maintain continuous integration and regression tests
            for z/OS mainframe software components, including Docker with Z Container
            Extensions, OpenShift Container Platform on Z, and z/OS REST APIs
            \item Orchestrated CI test automation infrastructure with Ansible and Jenkins to run
            health checks and test workloads across a series of on-premise mainframe
            test hardware
            \item Developed and ported test utilities with Metal C, z/OS High Level Assembler, and Golang to exploit z/OS low-level components and execute test utilities like Grafana k6 on native hardware
            \item Conduct daily monitoring and maintenance of mainframe test systems,
            including mounting filesystems, installing products including Conda, Nodejs,
            and OpenShift, and troubleshooting system dumps and traces in coordination
            with other developers and tester
        \end{itemize}
        
    }

    
    \expentry{Sept. 2019}
    {Nov. 2019}
    {Buildly}
    {Software Developer, Intern}
    {
        \begin{itemize}
          \setlength\itemsep{-0.5mm}
          \item Implemented bugfixes, unit tests, and refactored features for Buildly's Django backend in order to connect multiple microservices to a single endpoint, manage multiple databases, and incorporate self-documenting API specifications
          \item Maintained and refactored end-to-end test suite for frontend and backend with the Robot framework
          \item Communicated with CEO and CTO daily about product roadmap and development guidelines
        \end{itemize}
        
    }


    \expentry{Dec. 2018}
    {Feb. 2019}
    {UNICEF}
    {Software Developer, Intern}
    {

        \begin{itemize}{\leftmargin=0.5em \itemindent=0em}
          \setlength\itemsep{-0.5mm}
          \item Implemented utilites to \href{https://github.com/unicef/magicbox-download-roads}{retrieve road networks} for countries and \href{https://github.com/msradam/magicbox-site-routing}{compute distances} between coordinates in large geospatial datasets with machine-learning and networking libraries
          \item Optimized algorithm performance to improve computation speed by thousands of times across millions of geospatial points
          \item Provisioned a \href{https://hub.docker.com/r/msradam/magicbox-tools}{Docker image} bundled with these utilities and additional libraries for Magicbox developers and data scientists
        \end{itemize}
        
    }

    \expentry{June 2018}
    {August 2018}
    {Kurani Architecture}
    {Software Developer, Intern}
    {\begin{itemize}
          \setlength\itemsep{-0.5mm}
          \item Integrated Raspberry Pi sensor data and IBM Watson Machine Learning predictions in prototype IoT dashboards
          \item Prototyped learning feed with content scraped from TEDEd
          \item Pitched to and collaborated with CEO on IoT implementation in learning space architecture 
        \end{itemize}
    }


    \section{projects}


        \projectentry{July 2019}
            {Magicbox Geospatial Routing API}
            {Scikit-learn, Django, Docker, OpenAPI | \href{http://magicbox-routing.herokuapp.com}{magicbox-routing.herokuapp.com}}
            {\begin{itemize}
                \setlength\itemsep{-0.5mm}
                \item Implemented a geospatial computation microservice for UNICEF's Magicbox platform, with endpoints to compute distances between geographic coordinates and retrieve roads as graph networks 
            \end{itemize}
            }

        \projectentry{June 2019}
            {RESP: Responsive Emotional Support Protocols}
            {Node, Express, React, MongoDB | \href{http://resp-angelhack.herokuapp.com}{resp-angelhack.herokuapp.com}}
            {\begin{itemize}
                \setlength\itemsep{-0.5mm}
                \item Implemented RESTful backend server to check-in natural disaster survivors with Typeform webhooks, store them in a database, and query for information about nearby health facilities with the Healthsites.io API
                \item Designed a frontend with an embedded Typeform and databases for checked-in survivors and local health facilities
                \item Winner of the IBM Call for Code Challenge at AngelHack Manhattan
            \end{itemize}
            }
            
    \section{technical skills}

    \skillentry{languages}{Python, JavaScript (ES6+), Java, HTML5, CSS3, Bash}
    \skillentry{concepts}{RESTful API Design/Integration, MVC Architecture, Agile}
    \skillentry{frameworks}{Django, Flask, Express, React, Vue, Jest, Robot}
    \skillentry{tools}{Git, Docker, Kubernetes, Heroku, Google Cloud Platform (GCP), Amazon Web Services (AWS) }
    \skillentry{data}{PostgreSQL, GraphQL, MongoDB}
    \skillentry{testing}{Unit, Integration, End-to-End}

\end{document}
